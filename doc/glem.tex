\documentclass[a4paper]{article}
%
\usepackage{fontspec} %Font package
\usepackage{xunicode}
%
\usepackage{polyglossia}
\setmainlanguage{english}
\setotherlanguage[variant=ancient]{greek}
%\newfontfamily\greekfont[Script=Greek]{Galatia SIL}
\newfontfamily\greekfont[Script=Greek]{Linux Libertine O}
\setmainfont{Charis SIL}
\setmonofont[Scale=0.8]{Menlo}
%
\usepackage{xcolor}
\usepackage{titlesec}
\definecolor{MSBlue}{rgb}{.204,.353,.541}
\definecolor{MSLightBlue}{rgb}{.31,.506,.741}
%
% Define a new fontfamily for the subsubsection font
% Don't use \fontspec directly to change the font
% Set formats for each heading level
\titleformat*{\section}{\Large\bfseries\sffamily\color{MSBlue}}
\titleformat*{\subsection}{\large\bfseries\sffamily\color{MSBlue}}
\newfontfamily\subsubsectionfont[Color=MSLightBlue]{Hoefler Text}
\titleformat*{\subsubsection}{\bfseries\itshape\subsubsectionfont}
%
\usepackage{url}
\usepackage[verbose]{geometry}
\geometry{inner=3cm,outer=3cm,top=3cm,bottom=3cm}
\usepackage[style=british]{csquotes}
%
\usepackage{microtype}
\usepackage{setspace}
\setstretch{1.16}%onehalf is 1.25/1.213/1.241 for 10/11/12 pt
%
%\def\glem{\textgreek{Γλεμ}}
\def\glem{\textsc{glem}}
\def\conll{\textsc{conll}}
\def\proiel{\textsc{proiel}}
\def\frog{\textsc{frog}}
% ----
\newenvironment{varlist}[1]{%
\begin{list}{}{%
    \settowidth{\labelwidth}{\sf #1 }     %longest label length
%    \addtolength{\labelsep}{3ex}          %extra space between lbl/txt
    \setlength{\leftmargin}{\labelwidth}  %determine leftmargin
    \addtolength{\leftmargin}{\labelsep}  %determine leftmargin
    \setlength{\parsep}{0.5ex plus 0.2ex minus 0.2ex}
    \setlength{\itemsep}{0.3ex}
    \renewcommand{\makelabel}[1]{\sf ##1\hfill}}}%
{\end{list}}
%
\usepackage{tcolorbox}
\newtcolorbox[auto counter,number within=section]{pabox}[2][]{%
colframe=MSBlue,arc=4pt,outer arc=4pt,colback=black!5,top=4pt,bottom=4pt,boxsep=3mm,fonttitle=\bfseries, title=\thetcbcounter: #2,#1}
%
\renewcommand\arraystretch{1.2}% spacing in tables
%
\newlength{\pdfwidth}
%%\setlength{\pdfwidth}{1.0\textwidth}
\setlength{\pdfwidth}{0.45\textwidth}
\newlength{\halfpage}
\setlength{\halfpage}{0.5\textwidth}
\newlength{\halfpagefig}
\setlength{\halfpagefig}{0.5\textwidth}
\addtolength{\halfpagefig}{-0.25cm}
\newlength{\thirdpage}
\setlength{\thirdpage}{0.33\textwidth}
\newlength{\halfline}
\setlength{\halfline}{0.5\linewidth}
%{0.5\linewidth}
%
%\setlength{\parindent}{0pt}
\usepackage{parskip}
%
\renewcommand{\topfraction}{1.0}
\renewcommand{\bottomfraction}{1.0}
\renewcommand{\textfraction}{0.0}
% how hard latex should try to place floats together
\renewcommand{\floatpagefraction}{0.6}
%
%
%http://tex.stackexchange.com/questions/24766/long-underscore-in-latex
\newcommand{\uline}[1]{\rule[0pt]{#1}{0.4pt}}
%
%%http://tex.stackexchange.com/questions/299/how-to-get-long-texttt-sections-to-break
\newcommand*\justify{%
  \fontdimen2\font=0.4em% interword space
  \fontdimen3\font=0.2em% interword stretch
  \fontdimen4\font=0.1em% interword shrink
  \fontdimen7\font=0.1em% extra space
  \hyphenchar\font=`\-% allowing hyphenation
}
%
\newcommand{\fit}[1]{\noindent\resizebox{\linewidth}{!}{#1}}
\newcommand{\fcmp}[1]{\noindent\resizebox{\linewidth}{!}{\cmp{#1}}}
\newcommand{\scmp}[1]{\scalebox{0.96}{\cmp{#1}}}

% ========================================================================== %

\begin{document}

\selectlanguage{english}

\section*{Intro}

This is \glem{}, a lemmatiser.

%% copied from article

%1. what did we do
We present the lemmatizer that we developed for ancient Greek. Not only does it
disambiguate between multiple lemmas (with the same or different POS tags), but
it also creates new lemmas for unknown words.

%2. why did we do it?
Currently available lemmatizers for ancient Greek such as the one developed in
the CLTK \cite{johnson2014} cannot handle such cases.

%3. how did we do it
As the basis for the lemmatizer we used an existing memory-based learning tool,
Frog \cite{Frog2016}, that was originally developed for Dutch. We converted Frog
to work for ancient Greek. As the results of Frog on ancient Greek were rather
modest, we decided to create a smarter lemmatizer named \glem{} that combines a
lexicon look up with the memory-based tool Frog.

%4. what did we find
%We evaluate and compare the performance of \glem{} against the CLTK lemmatizer and
%the Frog lemmatizer and observe that \glem{} achieves the highest accuracy of 91\%
%on an unseen test corpus sample.

%5. what do we think that it means?
\glem{}'s look up component overcomes the difficulty of a relative small training
set in combination with a morphologically rich
language, %and leads to an improved precision
while the memory-based learning component enables \glem{} to handle unknown
words. % recall

\tcbset{width=\linewidth,arc=0pt,outer
  arc=0pt,colback=black!10,colframe=black!0,boxsep=1pt,left=0pt,right=0pt,top=4pt,bottom=4pt}

\section*{Usage}

\subsection*{Basic Usage}

Lemmatising a single file is done as follows.

\begin{tcolorbox}
\texttt{python3 glem.py -f TEST\_FILE}
\end{tcolorbox}

The input file is expected to contain plain greek text. This produces one output
file (and lots of output to the screen). The output file is named as follows:
\texttt{TEST\_FILE.lastrun.wlt.txt}

The \texttt{lastrun} marker is appended to the filename automatically, and can be changed with
the \texttt{-s} option. The following command will produce an output file
named \texttt{TEST\_FILE.run1.wlt.txt}.

\begin{tcolorbox}
\texttt{python3 glem.py -f TEST\_FILE -s run1}
\end{tcolorbox}

For each word in the input text, the output contains one line containing a
word-lemma-tag triplet. An example is shown below.

\begin{tcolorbox}
\textgreek{γενόμενα}\hspace{0.5cm}\textgreek{γίγνομαι}\hspace{0.5cm}\texttt{V--papmna-}
\end{tcolorbox}

If the \texttt{-S} option is added, two extra fields are added to the
output. The first one contains the entry as it is contained in the internal
lexicon in \glem{}, and the second one contains a textual representation of the
strategy used by the lemmatizer to reach the answer. The output file is named as
follows: \texttt{TEST\_FILE.lastrun.stats.txt}. An example is shown below.

\begin{tcolorbox}
\texttt{python3 glem.py -f TEST\_FILE -s run1 -S}
\end{tcolorbox}

And the output looks like this.

\begin{tcolorbox}
\selectlanguage{greek}ἱστορίης\hspace{0.5cm}ἱστορία
\selectlanguage{english}\hspace{0.5cm}\texttt{N--s---fg-}\hspace{0.5cm}
/\textgreek{ἱστορίης}/\textgreek{ἱστορία}/\texttt{N--s---fg-}/\texttt{2}/\texttt{greek\_Haudag}/\\
\strut\hfill\texttt{multi lemmas, no tag, highest frequency}
\end{tcolorbox}

\subsection*{Looking Up}

The following example shows how to use the \texttt{-w} option to look up a word
from the lexicons loaded by \glem{}.

\begin{tcolorbox}
\texttt{python3 glem.py -w τῶν}
\end{tcolorbox}

\begin{tcolorbox}
\begin{verbatim}
LOOKUP WORD τῶν
   τῶν, 14
     τῶν, ὁ, S--p---mg-,  1163 greek_Haudag
     τῶν, ὁ, S--p---qg-,   660 greek_Haudag
     τῶν, ὁ, S--p---ng-,   211 greek_Haudag
     τῶν, ὁ, S--p---fg-,   153 greek_Haudag
\end{verbatim}
\end{tcolorbox}

The next example shows how to use the \texttt{-l} option to look up a lemma from the
loaded lexicons.

\begin{tcolorbox}
\texttt{python3 glem.py -l Δάμαρις}
\end{tcolorbox}

\begin{tcolorbox}
\begin{verbatim}
LOOKUP LEMMA Δάμαρις
Δάμαρις
   Δάμαρις, Δάμαρις, N--s---fn-,     1 greek_Haudag
\end{verbatim}
\end{tcolorbox}

\section*{Frog}

\glem{} assumes \frog{} is available. It expects to be able to initialise
\frog{} with a template named \texttt{frog.ancientgreek.template}.

\par If \frog{} is unavailable, run \glem{} with the \texttt{-F} option to
disable it. 


\section*{Lexicon Files}

\glem{} reads three different lexicon files.

\texttt{greek\_Haudag.pcases.lemma.lex.rewrite\_new}. This file contains
word-lemma-tag-frequency information, one entry per line. The following example
shows two entries.

\begin{tcolorbox}
\selectlanguage{greek}αὐτός~~αὐτός~~\selectlanguage{english}\texttt{Pd-s---mn- 25}\\
  \selectlanguage{greek}αὐτός~~αὐτός~~\selectlanguage{english}\texttt{Pp3s---mn- 11}
\end{tcolorbox}

The second file is called \texttt{perseus-wlt.txt}. This file contains
word-lemma-tag information without the frequency information. The following
example shows two entries.

\begin{tcolorbox}
\selectlanguage{greek}ἀναπλέουσι~~ἀναπλέω~~\selectlanguage{english}\texttt{V-3ppia---}\\
\selectlanguage{greek}κατεπάγων~~κατεπάγω~~\selectlanguage{english}\texttt{V--sppamn-}
\end{tcolorbox}

The third file is an extra file containing punctuation. It is called
\texttt{extra-wlt.txt} and also contains word-lemma-tag entries. The following
example shows two entries from this file.

\begin{tcolorbox}
\texttt{(~~(~~punct}\\
\texttt{)~~)~~punct}
\end{tcolorbox}

\section*{All Options}

\glem{} accepts the following command line options.

\begin{varlist}{wwwwwwww}

\subsection*{Input and output}

\item[\texttt{-f} filename] The file (or files, wildcards are allowed) to
  be lemmatised.

\item[\texttt{-s} suffix] The output files will be given this suffix.

\item[\texttt{-S}] Outputs extra information consisting of the full entry from
  \glem{}'s dictionary plus a textual representation of the strategy used to
  come to an answer.

\item[\texttt{-W}] Assumes the test file contains word-lemma-tag data
  separated by whitespace. The first word of each line is lemmatised, and the
  specified lemma and tag are assumed to be correct and compared to \glem{}'s output.

\item[\texttt{-F}] The test file will be processed with \glem{}
  only, without using \frog{} for the unknown words. This means that unknown words remain unhandled.

\subsection*{Look-up}

\item[\texttt{-l} lemma] Looks up the specified lemma in
  the dictionary.

\item[\texttt{-w} word] Looks up the specified word in the dictionary.

%\vspace{0.5cm}
\subsection*{Lexicons}

\item[\texttt{-L} filename] Loads the specified lexicon file. 

\item[\texttt{-M} filename] Loads another "merged" file.

\item[\texttt{-E} filename] Loads another "extra" file. The
  default extra file contains punctuation.

\subsection*{Miscellaneous}

\item[\texttt{-v}] Extra verbose output to the screen.

\item[\texttt{-D}] Prints extra debug information to the screen.

\item[\texttt{-R}] Remove the token \texttt{ROOT} from the
  lines in the text file.

\end{varlist}

\section*{Strategy}

\glem{} uses the following strategy when lemmatizing a text.

\begin{pabox}{Lemmatiser Flowchart}
\begin{verbatim}
Check if word in dictionary.

If it is:
  1) If it has only one tag/lemma entry, return it.
     ("one lemma, same pos tag" / "one lemma, different pos tag")
  2) More than one tag/lemma entry: go through the tag/lemmas, and:
     a) if a lemma with a similar pos tag is found, return it.
        ("multiple lemmas, same pos tag, highest frequency" or
         "multi lemmas, same pos tag, other frequency")
     b) otherwise, return the most frequent tag/lemma.
        ("multi lemmas, different pos tag, highest frequency")
 
If it is not in the dictionary:
  1) Take Frog entry, and return it.
     ("Frog" / "Frog list")
  2) If this fails:
  return ("unknown")
\end{verbatim}
\end{pabox}

\begin{comment}
\begin{config}{e2}
    "MLDTHF" : "multi lemmas, no pos tag match, highest frequency", #DT=different tag
    "MLNTHF" : "multi lemmas, no tag, highest frequency",
    "MLSTHF" : "multi lemmas, pos tag match, and highest frequency",
    "MLNTHF" : "multi lemmas, no tag, highest frequency",
    "MLSTOF" : "multi lemmas, pos tag match, but other frequency",
    "MLNTOF" : "multi lemmas, no tag, other frequency",
    "OLDT"   : "one lemma, but different pos tag",
    "OLST"   : "one lemma, same pos tag",
    "OLNT"   : "one lemma, no tag",
    "FROG"   : "Frog lemma",
    "UNKNOWN": "unknown"
\end{config}
\end{comment}

\end{document}

